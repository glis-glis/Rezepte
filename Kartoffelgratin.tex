\documentclass[12pt]{article}
\usepackage[paper=a4paper,marginparwidth=0mm,marginparsep=7mm,margin=17.5mm,includemp]{geometry}
\usepackage{multirow}
\usepackage{lexend}

\title{Kartoffelgratin}
\author{Andreas}

\begin{document}
\maketitle
\thispagestyle{empty}

Gesamtzeit: 2h

\vspace*{1cm}

\begin{tabular}{l l}
  \hline & \\ [-1.5ex]
  1.2 kg & festkochende Kartoffeln \\
  5 dl & Rahm  \\
  1.5 TL & Salz  \\
  3 & Knoblauchzehen \\
  wenig & Pfeffer + Muskatnuss \\
  \hline
\end{tabular}

\section*{}
\begin{enumerate}
  \item Backofen auf 260°C vorwärmen
  \item Kartoffeln schälen und in ca. 5mm dicke Scheiben schneiden
  \item Gratin-Form mit Butter fetten
  \item Kartoffeln in die Form geben, Knoblauch darein pressen, Salz, Pfeffer und Muskatnuss dazu geben
  \item 2.5 dl Rahm dazu geben, gut vermischen
  \item Flachstreichen, restlichen Rahm dazu geben
  \item Gratin in Mitte des Backofen reintun, auf 240°C stellen
  \item Nach 10min Backofen auf 220°C stellen
  \item Nach 20min Backofen auf 180°C stellen
  \item Nach 40min ist der Gratin parat, auf 120° stellen bis zum Essen
\end{enumerate}


\end{document}
