\documentclass[12pt]{article}
\usepackage[paper=a4paper,marginparwidth=0mm,marginparsep=7mm,margin=17.5mm,includemp]{geometry}
\usepackage{multirow}
\usepackage{lexend}

\title{Zuckerbrötli}
\author{Andreas}

\begin{document}
\maketitle
\thispagestyle{empty}

\textbf{Gesamtzeit}: 3h

\vspace*{1cm}

\begin{tabular}{l l | l l}
  \multicolumn{2}{c|}{\textbf{Teig}} & \multicolumn{2}{c}{\textbf{Guss}} \\
  3l & Milch & 3 EL& Hagelzucker \\
  500g & Weissmehl &1 & Eigelb \\
  60g & Butter & + EL& Milch \\
  1.5 KL&  Salz && \\
  1 Pack&  Trockenhefe && \\
\end{tabular}

\section*{}
\begin{enumerate}
  \item Teig kneten. Kenwood: 5min Mischen, 4x Falten; 3x: 5min Kneten + 4x Falten
  \item Feuchtes Tuch über Teig und 30min ruhen
  \item 4x Teig falten
  \item 60-90min Teig ruhen
  \item Teig in 24 Kugeln aufteilen und auf Blech, 15min ruhen
  \item Backofen auf 220°C vorwärmen
  \item Hagelzucker in Geschirr reintun
  \item Eigelb und 1EL Milch verrühren, Teig-Kugeln damit bestreichen
  \item Teigkugeln in Hagelzucker-Geschirr reintunken
  \item Untere Hälfte im Backofen tun, Backofen auf 200°C stellen
  \item 20 min Backen
\end{enumerate}


\end{document}
