\documentclass[12pt]{article}
\usepackage[paper=a4paper,marginparwidth=0mm,marginparsep=7mm,margin=17.5mm,includemp]{geometry}
\usepackage{multirow}
\usepackage{lexend}

\title{Foccacia}
\author{Andreas}

\begin{document}
\maketitle
\thispagestyle{empty}

Gesamtzeit: 3h

\vspace*{1cm}

\begin{tabular}{l l | l l}
  \multicolumn{2}{c|}{Teig} & \multicolumn{2}{c}{Lake} \\
  \hline & \\ [-1.5ex]
  500 g & Weissmehl &3 EL& Olivenöl \\
  1.5 KL&  Salz & 3 EL& Wasser \\
  1 Pack&  Trockenhefe &5 Drehungen& Meersalz \\
  3.5 l & Wasser && \\
  3 EL&  Olivenöl && \\
  \hline
\end{tabular}

\section*{}
\begin{enumerate}
  \item Mehl + Salz + Trockenhefe in Schüssel tun und mit Löffel vermischen, danach Wasser und Öl dazugeben
  \item Teig kneten. Kenwood: 5min Mischen, 4x Falten; 3x: 5min Kneten + 4x Falten
  \item Feuchtes Tuch über Teig und 30min ruhen
  \item 4x Teig falten
  \item 60min Teig ruhen
  \item Auf Blech ausbreiten, ca. 2cm hoch
  \item Mit Finger Löcher in Teig machen: alle 4 Ecken + Mitte
  \item Teig auf Bleich, 15min ruhen
  \item Backofen auf 240°C vorwärmen
  \item Teig mit Lake bestreichen, Löcher mit Lake füllen
  \item Kleines Gefäss mit Wasser neben Foccacia auf Blech stellen
  \item Untere Hälfte im Backofen tun, Backofen auf 220°C stellen
  \item 15-20 min Backen (EW: 17 min)
  \item Rausnehmen, mit Olivenöl bestreichen und Meersalz darüber
\end{enumerate}


\end{document}
